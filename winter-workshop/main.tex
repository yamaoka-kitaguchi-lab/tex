\documentclass[twocolumn,a4paper]{ieicejsp}
\usepackage{ascmac}
\usepackage{graphicx}
\usepackage{url}
\usepackage{bm}
\usepackage{paralist}
\usepackage{nidanfloat}
\usepackage{fancybox}
\usepackage{float}
\usepackage{amsmath}
\usepackage{amssymb}
\usepackage{amsfonts}
\usepackage{pifont}
\usepackage{multirow}
\usepackage{comment}
\usepackage[dvipdfmx]{color}
\usepackage{caption}

\renewcommand{\baselinestretch}{1.0}  % 奥の手
\captionsetup[figure]{format=plain, labelformat=simple, labelsep=period, font=footnotesize}

\title{\bf なにかいい感じのタイトルが入る\\
\small{Something catchy headline here}}

\author{
    輪講 太郎 \\ Taro RINKOU
}
\affliate {
    東京工業大学 \\ Tokyo Institute of Technology
}

\begin{document}
\maketitle

\section{はじめに}
冬ゼミテンプレートです.
図~\ref{fig:tech-chan}とか文献\cite{myself}とかは適宜置き換えてください.

\begin{figure}[!t]
  \centering
  \includegraphics[width=60mm]{tech-chan.png}
  \caption{工大祭公式マスコットキャラクター}
  \label{fig:tech-chan}
\end{figure}

\begin{thebibliography}{9}
\footnotesize
\bibitem{myself}
輪講 太郎, 北口 善明, 山岡 克式, ``情報通信ネットワークに関する研究,'' 信学技報, vol. 120, no. 293, IN2020-99, pp. 13-18, 2020年12月.
\end{thebibliography}

\end{document}
